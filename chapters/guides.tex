\chapter{Guides}

This is it - the place 80\% of readers most likely will have skipped to in order to check out what to aim for, what to do, what to not do. And I'm more than happy to provide. However, before I can just go ahead and drop you a cheat sheet to check out, I feel that we need to define one last thing - the stages of the game.

The community surrounding C:DDA generally talks in broad terms as to what to do and aim for - The early game, the mid game and the late game. But what defines these points? While I cannot speak for everyone, I will give out my own definition of these terms and what the appropriate goals should be in these. For reference - the starting scenario as well as profession will have a huge impact on your goals and will shift dramatically. Depending on how lengthy each section will be, I might be inclined to write quickguides for the appropriate and more commonly used spawns. For this general assumption we will take the standard 'Evacuee' Scenario as well as the standard 'Survivor' class.

So, to talk about those stages...

\paragraph{The Early Game:}
It's' the initial struggle and you require the most basic needs, this means water, food, sleep in that order. Some initial gear besides what you spawned with would also be nice, maybe some extra clothing and means to haul items from a to b.

As you will be struggling in one certain location for a long time, a shelter that would be your makeshift base of operations would also not go amiss.

Once you have started settling in and are having no trouble with the intermediate enemies the game throws at you, you can officially say that you managed to get past the early game

\paragraph{The Mid Game:}
Getting established in terms of a true base, either mobile or stationary is the main driving force here. For gear that you will loot, you are most likely looking for medium to high caliber firearms, or state-of-the-art bows and crossbows. Expanding your base and vehicle, fortifying either of those with appropriate defensive measures is also a point you want to work on. You will now also be looking to stock up on supplies for the coming seasons, namely winter. General enemies like zombies are no problem to deal with, except for a hulk maybe. This means deaths occurring at this stage are either due to enemies having superior firepower, you being a careless git or overwhelming numbers whittling you down.

\paragraph{The Late Game:}
This is where it's at. You are probably the most powerful being on this planet (to your knowledge), and without cheap tricks, it'll also stay this way. Your base - either mobile or stationary - is fortified to a degree that'll put Fort Knox to shame and the firepower you carry is trying to find an appropriate match. You basically reach this stage by either having a brutal deathmobile that kills everything for you or by having Power Armor and the weaponry to be a rolling fortress yourself. Supply stocks are filled to the brim and you are now actively looking for challenges. This is where you either retire due to nothing posing a threat or you dying because you found something that did pose a threat and was in no rush to show its superiority to you. Make sure to tell us on Discord how you died.

With that out of the way, the following pages will contain a page of information on their respective first pages carrying information what you should aim for, what is important to get out of the way and what would be considered 'Luxury'.

\section{The Early Game}
Immediate needs:
-Water
-Food
-Shelter
-initial tools

Important gear to collect/craft:
-appropriate clothing
	-tolerable levels of Encumbrance
-Temperature
-Armor
-Volume
-Tools
	-Pot/Frying Pan or equivalent
-Screwdriver/-Set
-Hammer
-Hacksaw
-Wrench
-Soldering Iron
-Weapons (according to taste)
	-Bash: Baseball Bat, Nail Bat, Aluminum Bat or better
	-Cut: Makeshift Machete, Machete, or better
	-Pierce: Combat Knife, Copper Spear or better

Luxury:
-Tools
	-Toolbox
	-Welder/Acetylene Torch + Welding goggles/PBA Mask
-Meds
	-Antiseptic (Either liquid or powder)
	-Painkillers
	-Antibiotics
-Weapons
	-Shotgun of any kind
	-Rifle of medium/high calibre (.45 ACP minimum)
	-powerful melee weapon (Katana, Steel Spear, Mace etc.)
-Shopping Cart or similar
-Any sort of books that raise skills/contain recipes
Goal: Survive

First things first - check the Terminal, select 'Contact Us' to get a marker to the closest refugee center as well as (hopefully) the main road to follow just to uncover more of the map. Back to your needs:

How do you achieve all of this? Water first - Either loot Houses/Grocery Stores, or if towns are too overrun to enter, find a suitable item that has boiling quality of 1. This can range from Glass containers (jar, 3L jar, bottle) to a tin or aluminum can. If there are plentiful Forest tiles nearby, you can forage the underbushes by [e]xamining them and maybe manage to find trash, to which those containers all count. As a byproduct you will also find lots of vegetables (Note: as of the recent rebalances in experimental, veggies barely are worth picking up, making forest spawns more difficult) and eggs that will serve as an initial food supply, ticking off two things at once (a frying pan/pot or stone pot is however required to process those into truly safe foods).

The Evacuee Shelter makes for a decent shelter to begin with, since it has a permanent light source (Terminal), lots of crafting resources (Benches, Curtains) and a basement (fixed temperature to store your food in so it won't freeze/thaw and subsequencially rot).

Some of the tools on the list can be crafted early, at least makeshift or crude versions of it. Most of those require Fabrication and Survival in the ranges of 1-2 each, so once water and food are set and you can't go looting, do some crafting instead.

A decent weapon of your choice should be mandatory before entering combat. A Nailboard/Makeshift Crowbar will do for bashing weapons, a wooden spear is pretty much your only option for pierce (2 fab/1 survival) and cutting users can go with the crude sword (requiring the 2-by-sword, also craftable), or a makeshift machete if they can find and disassemble a lawnmower and some duct tape.

If you have had the luck/bad luck to not see any cities in the immediate surroundings of your starting shelter, once you are stabilized, that would be the perfect opportunity to venture out, follow the road and look for a city to loot.

The bigger tools on the list (Hacksaw, Wrench, Screwdriver set etc.) are far off of your crafting, as they require a forge and appropriate sub-tools to be created, so you are better off finding those inside mechanics-related buildings like Bike Shops and Garages or even in the trunks of vehicles, which you should check regardless as they can contain many useful items.

Clothing you either find in a clothing store or by looting houses. Make sure those are not a poor fit, as the encumbrance penalty for those is quite high compared to their fitting forms. Clothing to look out for are (but are not limited to): a (leather-)Backpack, Duster/Trenchcoat, Cargo Pants, any sort of helmet, sunglasses (for glare protection 1), if you can manage a filter mask/gas mask, Knee/Arm Pads. (Steel-toed)- boots. Basically it should be sturdy, preferably made out of leather (higher armor), keep you relatively warm (undershirt, long underwear, hoodie even) and have a decent amount of pockets. Make sure you are not too over encumbered on your torso/legs as this will hinder you in dodging and running away. Nevertheless, if clothing is difficult to obtain, remember that tailoring them yourself is also an option, though this should be postponed till you can actually support yourself with water and food.

If you manage to find a shopping cart/wheelbarrow, consider yourself lucky. You can grab these vehicles using the [G] command and drag them alongside you, dropping items into them using the [D] + NumPad direction command.

Books you come across should be weighed carefully - morale books are totally unimportant and can either be turned into paper or left behind for later, not like they are rare. Anything with recipes in it should be brought with you, especially first aid, electronics and mechanics books. In a pinch, don't hesitate to drop that book in order to pick up something of more immediate importance if you struggle to survive. You may always be able to take a track back to pick up that leftover stuff, but the most important matters should be resolved first.

Combat in this stage is relatively dangerous and should be avoided if possible, or dealt with carefully - remember to utilize your surroundings: Create choke points to deal with 1 Z at a time, prioritize high threat targets (Spitter/Acidic/Shocker/Shrieker), if worse comes to worst, fire can save your bacon. (not indoors, as this'll torch the building, or maybe that's the plan - beware of noise) inevitably though, you will have to deal with enemies one way or another. Bite wounds are difficult to clean unless you happen to stumble over a first aid kit, which will last you for a long while. This is also the reason I recommend piercing weapons - especially spears - as their range will make it comfortable to deal with zombies.

If you manage to get settled in, are reasonably equipped and have a couple days' worth of food and a steady supply of water sitting alongside your tools, consider the early game struggle beat.

\section{The Mid Game}

Immediate needs:
-stronger gear
-vehicle tools
-drugs/meds
-food stockpile
-a game plan moving forward

Important gear to collect/craft:
-Better Weapons:
	-Bash: Expandable Baton/PR-24 Baton
	-Cut: Any medieval Sword (Broad-, Longsword)
	-Pierce: Steel Spear/Awl Pike/Estoc
-Firearms:
	-shotgun loaded with 00 Shot (if not obtained earlier)
	-rifle of medium calibre or upwards (.223, .308)
-Tools:
	-Acetylene Torch/Welder (really important now)
	-Forge of some kind (depending on your preferences)
	-Food processing tools (Smoker, Processor, Dehydrator)
	-vehicle tools (Jack, Lifting tools)
	-utility tools (Bolt Cutters, Drill, Jackhammer)
-Clothing:
	-Leather gear or armored equipment

Luxury:
-A functioning car fitting for your needs.
-CBMs stockpiled up
-Martial Arts / Weapon Arts
-High-Power firearms (M249, M60, M107A)

Goals:
-get a vehicle/base up and running
-improve your gear

As you can see, the list is already shrinking - your most pressing needs are now advancements that are not crucial to survival, but to make survival even easier than it is. Gear currently equipped can now be interchanged for gear that you feel is better for your survival needs, like stronger defense in combat or more storage capabilities and you will sooner rather than later require a set of wheels to drive you around, as walking longer distances takes considerable time besides limiting what you can bring along.

Stronger gear can be anything, it all is relative to your already existing gear and how badly it is damaged. Mostly maintaining the current gear until you run into a nice drop is your mainstay, unless you found a recipe book and get to crafting, which takes its time. At this point, checking Museums/Antique Stores/Pawn Shops is now a totally viable option, as the randomly encountered set of medieval armor or weapon can really swing the game around. Just make sure it is not a fake (read the description/check the material). Why didn't we check those buildings out earlier? Well, at least the first and last suggestion always come with an alarm and are also most of the time locked, meaning you more often than not have to trigger the alarm to get the contents of the building.

While exploring towns, check out the different vehicles and assess the damage to them - are they in relatively drivable condition? Which of the vehicles would make for a decent mobile base? How difficult will it be to get the vehicle back up running? And do not forget to check the inside of vehicles for items. A Scissor/Bottle Jack is pretty much mandatory to fix tires. Something with lifting quality wouldn't go amiss either. But in the chance you did not find a telescopic cantilever or boom crane, you could just weld a forklift arm onto a frame and drag that along, as a forklift arm has lifting quality 1.

Pr0-Tip: Vehicles that make for great starting points for a Mobile Base are RV/Meth Lab, Firefighter Truck, Low-end Cube Van, Transport Truck.

But what about that Acetylene torch or Welder? Well, those are rare finds in mechanics-related buildings, which is why you are most likely better off making yourself a makeshift welder (requires 3 in mechanics). The Skill grind for it can be grating if you haven't managed to find the book 'Under the Hood', but more often than not, it may just boil down to you grinding up mechanics so you can make that welder. The goggles required also can be a nuisance to obtain. You either require Welding Goggles, Eclipse Glasses or a Firefighter PBA Mask.

Welding goggles can be made with mechanics 2. It is, however difficult to get a hold of all the components, as you need to disassemble lots of sunglasses to obtain the tinted lenses. How do you get the heating elements and the wire to make the welder though? Well, heating elements are part of ovens (deconstruct furniture) and dryers and copper wire comes from furniture like terminals, dryers, washing machines as well as engine blocks, which you can remove with a wrench and something with lifting quality (or the appropriate strength)

Clothing and Armor is another thing you find yourself wondering - you may or may not be taking down zombies quickly at this point, but taking less damage in the process can't be bad. So you have either the option of looking for better clothing specifically on your loot runs, or, more likely, are forced into tailoring gear yourself. Good thing that sheets are plentiful from curtains, as are long strings, providing you with all the thread and rags you could ever think of. Leather might be a bit more difficult to obtain, though what you could do is dismantle car seats, as they are apparently all made out of leather in the US. A bus would provide you with lots and lots of leather to work with, alongside pipes and springs which may come in handy for crafting. At this stage, decent Clothing options are, but again not just limited to:

Leather Armor Helmet/Army Helmet/Riot Helm
Leather Armor Boots
Leather Body Armor
Pair of metal Arm guards (if your arms are naked, for example from wearing the body armor)
Leather Armor Gauntlets
Pair of tactical gloves
Leather Duster (if you couldn't find one)
Armored Leather Jacket/Trenchcoat + Leather Pants
MBR Vest (Kevlar)/Kevlar Vest

But what about those weapons? It may be smart to develop your skills in different aspects, like learning the other melee weapon classes to a certain degree, depending on what you find out in the field. Same goes for ranged combat - some ammo types are generally too weak to use in real combat, most notably .22LR. Which means it is a safe ammo type to use for gaining levels on slow moving targets. Developing Archery and Rifles by use of Crossbows/Bows is also viable, since you are now settled in with proper ways to kill most common enemies. These paths allow for pretty powerful Mid-to-Late Game ranged combat if trained to the appropriate point (Archery 5 in particular grants a huge power spike - this is however, subject to change as darktoes at the CDDA Discord hinted at a redux of archery entirely).

Finding better melee weapons is gonna be a weird coinflip situation at this point and you are most likely forced to forge yourself better weapons in order to get the most bang for your buck as Swords, Maces and Spears/Awl Pikes are generally speaking only found in Museums, Antique Stores or Mansions. This would however require a forge and appropriate amounts of fuel (either battery or charcoal), a boatload of materials to start out with and most notably - time. Several forging recipes take up many hours of crafting to get you something in return, which is why you want to be situated pretty well before potentially working on that. If you however do find yourself a book that contains weapon recipes for those types of weapons, consider yourself lucky and consider the option of setting a couple days aside to grind up the skills to appropriate levels in order to obtain a better weapon.

But how do you stock up on resources? Especially food is important, preferably long lasting, as cans of food only get you so far. You will require some form of food preservation.

This can be achieved in many different ways - you could build yourself a Smoking Rack and a subsequent Charcoal Kiln using the construction menu to preserve meat butchered from the wildlife or vegetables found while foraging, making them last longer. If you have the batteries to spare, a food dehydrator has the same effect and can be crafted if you happen to stumble across the recipe in electronics related books, or found in a household. A charcoal smoker is pretty much a charcoal-powered version of the food dehydrator, except you won't have the functionality of a smoking rack, in that you can load it with meat and smoke food in it, so it only works as a tool for crafting. While not a necessity, a food processor makes for a great item to turn various crops and plants you find in the wild into flour which is a great way to preserve it for very long stretches of time. A canning pot is as of recently also required (alongside cooking 4) if you wish to preserve food for eternity by using glass jars. This however has the added benefit of the food not just staying fresh forever as long as it is still sealed, but also tasting good, increasing your mood.

Once you are able to finally call a sufficient vehicle your own, managed to obtain a nice batch of tools and weapons/firearms and have stocked up on food, you are able to hit the road, explore, loot, explore more, run into death you didn't know existed, you can enter the Late Game.

\section{The Late Game}

Immediate needs:
-none

Important Gear to collect:
-variety of CBMs
-lategame Weapons
-Convenience tools / Vehicular Toolstations
-late game recipe books

Luxury:
-Power Armor
-laser weapons
-fun

Goals:
-challenge the different late game encounters
-not getting bored

So as you can see now, there isn't much else left to do at this stage. You managed to get to a point where most normal enemies won't even be able to deal damage to you and even earlier threats are easily dispatched.

So, how do you deal with the remaining challenges up ahead? Preparation. Not many enemies are threatening to you at this point and it's more of a chore than a difficulty to clean out towns, fight off enemies and loot, but some strong enemies still lie ahead. Improving your gear and preparing for the worst is the answer. It is all about improving your gear and always has been, yet there's only so much room what areas you can improve by your skills alone.

But let's take it from the top:
As already explained, CBMs can be a sound alternative to boosting your characters power. Collecting them is not that difficult, and a crafty survivor may do it already at the very early stages of the game, by dissecting various enemies.

Pr0-Tip: Following enemies that you can encounter at random early on contain CBMs - Shocker Zombie/Brute, Zombie Scientist, Zombie Bio-Operator and Zombie Technicians.

Yet, installing those requires skill, anesthetics and most importantly - an AutoDoc. This is a structure most commonly found inside Science Labs and Hospitals, rarely inside basements. The one inside a Hospital seems to be the most feasible to get to, as hospitals are pretty open and therefore approachable from different angles, yet this is actually far from the truth.

Usually hospitals are overrun with zombies on the inside, and zombies inside a building tend to thrash around, destroying walls and furniture, which also happens to be what the AutoDoc is considered to be. More often than not the sheer amount of zombies inside a Hospital will have them wrecked beyond recognition and just to loot the place you will most likely require a shovel and many hours of work.

So you either check several basements, hoping for one of them to be a Basement that contains the AutoDoc Station, which is hidden inside the basement behind a door that in turn is hidden behind lockers in a different room, or risk entering a Laboratory. So unless you are observant you will most likely miss the station, but not only that - basements are in and of themselves a risk to take. At the Late Game stage, not many things can even damage you, which is why I only mention them here. However, entering a basement just to find it filled to the brim with Spiders, a Sewer Gator, Dark Wyrms or Zombies can be a quick way to lose a character, so you might want to make sure that you can take the risk.

Pr0-Tip: (Works only with Experimental Z levels enabled) If you hear talk lines that refer to robotic enemies inside a town while not seeing something like a sewer entry, you can be pretty certain that one of the nearby buildings' downstairs contain a bionic basement, as there is an Insane Cyborg enemy inside them guaranteed, creating those voice lines - or better yet, a lab entrance

But what about the Science Labs?

They make for a great challenge to undertake and come with great Endgame rewards, yet they can be not only dangerous, but outright deadly to enter for any character that is not prepared for what lies ahead.

To enter a Lab, there's several different approaches. For once, the good ol' fashioned Science ID card. The Card reader in front usually is functional and swiping the ID (which will consume it in the process) will open the doors. If you lack the ID but managed to install a Finger Hack CBM or have a USB Drive with HackPro on you, you can also attempt to hack the reader, which can have some negatives on a fail, but is a secondary way of bypassing terminals and readers. The third and most likely approach is brute force - a Jackhammer has no trouble mining through the walls in order to break and enter. Beware, however as this will lead you right into the turret that is placed behind the doorway. Using an ID or hacking the Reader will despawn this turret, while brute force will not.

In order to not spoil the Science Lab too much for you, this advice on different ways to enter it is the only one I will provide, now it's all up to you, and tread carefully.

With that in mind, what else should you work on to make life on the road and as the most powerful being more convenient? Well, Tool Stations for vehicles wouldn't go amiss. A Welding Rig, Kitchen Unit/Chemistry Lab and maybe a fridge. A vehicle forge and the FOODCO Kitchen Buddy all make for great additions to a mobile base. Those can either be found in the appropriate vehicles or, more likely - be crafted by yourself if you happen to have the appropriate books. But how do you protect these valuable stations from being smacked up by zombies while you are out, looting? Either by armoring up and providing your vehicle with a nice batch of different armor parts, or by using turrets.

Turrets require a mount to be installed upon and take up the 'On Roof' space of a vehicle, so you can't install them over your solar panels in hopes to minimize space used. Not just that, but the mount itself does nothing, it only allows you to install weapons onto that position, those include LMGs of various kinds, Laser weapons and other potentially deadly vehicular-mounted weapons. To fire such a weapon, you require a dashboard/electronics control unit on your driver seat to set the aiming/firing mode of turrets that support this feature. Pretty much all LMGs, the Browning HMG and laser weapons all can function as autonomous turrets, firing on their own. Some weapons however to have a control unit under the position of said turret in order to aim and fire them manually, so make sure to check the vehicles options menu to see if you can aim and fire a turret from your drivers seat, or if you have to do work manually.

So, what about this End Game gear that I was talking about? Well, for once, there are buildings we haven't even talked about, either because they didn't seem all too valuable in the earlier stages of the game, or are too heavily guarded to bother with.

Not only labs provide a nice batch of CBMs and Late Game gear that is ready for the taking, but there's also some heavy loot to be found right in the middle of cities - namely inside Bank Vaults.

However, as of the recent versions, banks do seem to come packed with security bots, which are more than willing to turn anybody heading too close to them into swiss cheese, which is why I avoided talking about them. Not just that - but any building that is secured with an alarm (Museum, Pawn Shop, Banks to name a few) and which is broken into (destroying a window, breaking the door etc.) will cause an Eyebot to spawn. They are not too terrible on their own, yet have the ability to call in reinforcements. This can quickly overwhelm you and end your life, so you might want to exercise caution when attempting some breaking and entering.

In addition, if you wish to crack safes that you come across quietly, you will require either a stethoscope or the Enhanced Hearing CBM. Stethoscopes can be found inside ambulances, the CBM, well, we talked about that earlier. Not only do bank vaults rarely contain some really decent recipe books regarding items that require Plutonium, their safes also contain pieces of Power Armor, the best piece of gear you can wear. It makes you pretty much impervious to normal attacks or small/medium caliber gunfire and shrapnel, depending on range. With that worn you are pretty much a walking tank, ready to wreck the world. That is, if you can operate it.

What else is there to work towards? Well, bigger guns if lasers aren't your style. Gun stores contain all different kinds of calibres, though the biggest and baddest firearms usually come from military bunkers or the barracks inside Science Labs, so you have something to look out for while exploring this content.

Or maybe you are tired of having to fight off the enemies and wish to take the fight to them? Then look no further, as there are several Overmap Specials that you can challenge for a good old fight. All the Fungal Structures should be avoided until you are at this stage unless you know precisely what you are doing - a fungal infection is pretty much a game-ender, despite is not technically killing you if you have no way to cure it.

Areas of interest that you may find more or less challenging at this stage, according to your gear are the following:

Triffid Grove
Strange Temple
Mine
Anthill
Sulphurous Anthill
All Fungal Overmap specials

\section{Wilderness Start}

Immediate Needs:
	-Shelter!
	-Water
	-Food

Important Gear to collect:
	-boiling quality
	-starting tools
	-Building Materials
	-clothing

Luxury:
	-finding proper tools
	-finding a crash site / wreckage
	-making / Finding a cart

As you can see, the priorities shift heavily with the Wilderness spawn of any character, except maybe the hardened survivor, who is clothed beyond belief. Why is shelter so important? Well, temperature kills, and it kills you quickly - you can die within the first 2 days, simple due to freezing your head to death.

This spawn does heavily shift the games' early difficulty around, depending on how much contact with civilization you wanna have. If you want to avoid cities all together, you will struggle with clothing and many electronics/mechanics related recipes due to lacking resources.

The most immediate need is therefore shelter - remember the construction key [*], in order to set one up real quick - good locations are near a river, near a forest tile with a water tile close by, a pond - basically anything with water in walking, or preferably - in crafting range.

Water is the next pressing need, as the shelter should for the first days of spring keep you at (chilly) at best, which is more than survivable with a couple fires (just make sure to contain them with deep pits). So if you didn't spawn with any sort of container or tool that is able to boil water, this is your next task - finding/making one. Either find a tin/aluminum can, glass bottle/jar/flask and go nuts, or make a stone pot, which however requires Survival 2 and Cooking 1.

If you only wanted a quick glimpse of how to make it through a forest on a wilderness spawn, yet want to head to the nearest town, this is pretty much it for you, as you are now more or less settled to go out and hunt for towns to loot to your heart's content.

Water helps out a long way, but only is so much, food on the other hand can easily be found by foraging, hunting small to medium game and this also has the added benefits of providing you with containers for boiling, in the form of sealed stomachs, as well as potential pelts, since clothing is sparse inside forests.

On that note - if you manage to find a netherworld spawn with several dead Humans, consider yourself lucky, clothing only comes in so many forms, and finding any piece of extra clothing is a blessing if you decide to live out in the forests, minimizing contact with civilization.
Same goes for heli-crash sites, as they have a boatload of metal and can easily fuel a forge, depending on size. And Rocks aren't all too uncommon inside forests, making a forge with a kiln set aside all the less troublesome.

Food comes in a huge variety inside forests and more often than not - it comes straight towards you. You might wish to take a bit of time off to designate some deep pits (don't bother spiking them) in the shape of your base, or even a small house, like a 5x5 around your shelter. As deep pits will come in real handy once you want to upgrade your makeshift shelter to a real building with log walls, which are more than easily obtained in a forest and surprisingly sturdy.

The forests are only as hard as you make them for yourself, you might need to be a bit more creative with weapons, as bows make for a decent weapon (if you get a bunch of feathers for the fletching), as does a sling. A wooden spear for melee users and you can go conquer the wilds.