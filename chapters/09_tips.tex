\chapter{Tips and Tricks, FAQ}

So, you either skipped to this spot or decided to read through the entire Guide and are now interested in some quick tips for certain things. Or maybe you wonder if a question you might be asking yourself will be answered here? We shall see. For those wondering why absolute base questions pop up here, even though they have been answered - this is for those absolute beginners that want to experience the game on their own, but are wondering why certain things don't work out for them as they expect.

\textbf{Q:} How to I raise (insert skill here) from (insert level here)?\\
\textbf{A:} Either by finding books (unlikely to find the exact book you are looking for) or by performing the appropriate action. Combat skills raise naturally and you have to chin it up for the most part, as books only get you so far, but for anything crafting related, the biggest tip I can give to you is this - use the recipe search [f] while inside the crafting window.

By searching using a parameter you can easily find what you should be crafting in order to specifically raise a skill - So if you require fabrications, do 's: fab' in order to find a recipe that has the skill fabrications required and search through the list. Works with any skillname, either partly or in full. Remember that there aren't recipes for each level that are automatically learned, this is meant to incentivise you looking for books or learning the skill out in the wild.

\textbf{Q:} How do I raise mechanics from 0?\\
\textbf{A:} The classic one, since mechanics from 0 is somewhat more difficult to raise, as you won't find recipes relating to it in an unmodded game, or with some side-skills taken at the very start.

My general suggestion - make 4 wooden frames, have 40 nails around and create 2 vehicles (construction menu), after starting the vehicle, install a wooden box into the 1 tile (make sure you have high focus for this, otherwise it'll take more)

This usually gets you to mechanics 1, at which point you can uninstall/install smaller parts from existing vehicles using a screwdriver.

\textbf{Q:} Why do my shots keep missing even at close range?\\
\textbf{A:} Make sure to aim by waiting [.] or use one of the appropriate aiming options in the firing menu. [a]im, [c]areful aim or [p]recise aim respectively.

\textbf{Q:} How do I cook food and heat up existing food?\\
\textbf{A:} use the crafting menu [\&] and select what you wish to create. To heat up food, you require a tool you need to activate (Frying Pan/Pot/Stone pot etc.) and either a fire or tool that can work for heating (Hotplate, Mess Kit etc.)

\textbf{Q:} Why am I sick?\\
\textbf{A:} General sicknesses like the common cold of flu just happen, unless you wear a gas mask or similar constantly, avoiding those is gonna be next to impossible. If you got poisoned or food poisoning, this happened because the food you decided to throw into your system wasn't meant for eating raw. Some enemies can also infect you with disease-like status effects.

\textbf{Q:} How do I run from zombies?\\
\textbf{A:} Press ['] to toggle between walking and sprinting. Also make sure to break line of sight, even though zombies can track you using scent, their vision is the most important sense.

\textbf{Q:} How do I loot towns?\\
\textbf{A:} be creative! Use fire to torch an already looted building to attract zombies to burn to death, raid at night (carries its' own risks), or just run through it with a pair of rollerblades while dragging a cart behind you.

\textbf{Q:} Why can't I hit stuff in melee?\\
\textbf{A:} Either you lack the skill in your weapon category or (more likely) you are too over encumbered with your gear. Encumbrance is king when it comes to combat and you will have a hard time hitting an enemy with 2 backpacks and 2 makeshift slings worn. So feel free to read the section 5.3.2 Temperature and Clothing to get a general idea of how clothing affects you, or 5.3.5 Combat to get a better grasp of combat.

\textbf{Q:} How do I train my skills?\\
\textbf{A:} Not only was this answered as the very first question, but again. Use them! Every action performed that is linked to a certain skill will increase it naturally, or if you happen to find a book, read that.

\textbf{Q:} Something keeps hurting me but I'm not being attacked or anything?\\
\textbf{A:} You most likely have caught a parasite by eating raw meat. Curing that requires some antiparasitic drugs.

\textbf{Q:} I'm infected, what do I do?\\
\textbf{A:} Pray to whatever higher being that you will survive. If you are however an atheist, your best, and only option at this point are to find some Antibiotics or Royal Jelly to cure the infection. Atreyupan is only a prophylactic measure and can't cure infections, only slow them down, prolonging your time window from 24hrs to 48hrs. Your best option is to not let deep bite wounds get to the point of being infected.

\textbf{Q:} How do I stop getting infected?\\
\textbf{A:} Any deep bite that you obtain will turn your limb from the normal white Name (i.e. TORSO: \textbar\textbar\textbar\textbar\textbackslash) to a blue coloured name. This can also be seen in character overview [@] under status effects (the bottom right part of the window). When you are at this stage, a deep bite wound can either be cleaned using any form of disinfectant (disinfectant, makeshift disinfectant, hydrogen peroxide) on said limb, or it can heal naturally over time. The chance for a bite healing on its own is based on your hidden health stat, so eating and living healthily has its benefits, still, your best option is to carry around a small bottle of disinfectant for on the spot cleaning of wounds.

\textbf{Q:} Where can I get a long-term water supply?\\
\textbf{A:} Forests contain water tiles you can fill containers in, if those aren't an option - swamps either have salt- or freshwater in them, but usually always contain both somewhere so make sure to check every water tile, even connected ones. If nothing like that works, you can just use a funnel on an open tile and place a water container below (drop it on the same tile).

\textbf{Q:} How do I control a vehicle?e\\
\textbf{A:} [\^{}] on the driver's seat. If you wish to learn more about vehicles in general, check section 5.3.11 - Vehicles.

\textbf{Q:} Why do I keep dying in combat?\\
\textbf{A:} First of all - Welcome to Cataclysm, hope you'll enjoy your stay. But to answer the question - combat has several factors that can easily decide a battle on their own (Pain, Encumbrance, difficulty of Enemy etc.), please feel free to read about combat in the appropriate section.

\textbf{Q:} I just started out but idk what to do.\\
\textbf{A:} This ain't a question but rejoice nonetheless! Read through this tutorial in its entirety please.

\paragraph{More unassorted tips:}
\begin{itemize}
\item Make a sling early game for ranged combat and fire pebbles at enemies, it is surprisingly strong against unarmored targets. Anything that is armored will give you trouble thou.
\item If you can't find a shopping cart, make your own by using 2 wooden frames (just a vehicle frame + wooden box on the same tile), will create noise when dragged, but it is storage.
\item Beware that most buildings that would have an alarm system in real life come with an alarm system in the game, not only attracting zombies nearby, but calling in eyebots. (houses excluded)
\item Whats unhealthy in real life is most likely also unhealthy in game.
\item Pick your fights carefully, Just because you can beat 3 zombies in a group, doesn't mean it is the same with 3 Zombies + 1 Spitter Zombie
\item You need lifting quality for a vehicle part? Just weld a forklift arm (requires steel frame) on to any frame and drag that along. How you can weld a metal piece onto a frame made out of wood, I have no clue.
\item Always keep some medical supplies handy - having to run back to your base because you forgot antiseptic is bad enough as is, but bleeding to death because you didn't want to carry 3 bandages with you is your own fault.
\item Peek around building corners using [X] - enemies will most likely not notice you doing so.
\item The game runs only when you decide to. Take your time and assess your situation to check every possibility.
\item While prioritizing what loot is important is a good thing, make sure to get a general idea of what is used as a component for crafts and where to obtain it.
\item On that note, read the item descriptions and its additional modifiers carefully! Countless newer players used fake medieval weapons only to wonder why they sucked so much using it.
\item Deconstruct furniture (construction menu) instead of [s]mashing it for getting the most resources out of it.
\item Electronic components can be obtained from Terminals, Arcade Machines, Washing Machines/Dryers. Ovens carry Heating Elements.
\item If you have trouble keeping your food fresh for long, try building a Root Cellar or finding a frozen Science Lab, which will cool your food below freezing point.
\item Need a Distraction? Anything creating noise is great, like a two-way radio, firecrackers or a noisemaker to draw zombies from one location to another - works especially great at night so just activate and throw
\item Vehicles are the solution to most of your problems - Locked doors? Ram it. Zombie Hulk? Ram it. Low on food? Insert vehicle at high speed into edible animal.
\item Turrets and various kinds of robots are actually your friends. Not only are they pretty well armored, they also come with several firearms equipped to deal with the nasty hordes of the apocalypse. Just drag zombies into the more tame ones (like eyebots) while trying to get away, make sure to break line of sight and to NOT walk into those robots. The last thing you want is you riddled with bullet holes from a tank drone.
\item Same works for animals - if you can manage to position yourself between a moose and some zombies, by running ['] quickly past a Moose or Bear in a circle motion, you can make the zombies take a path that is a smaller circle and way more likely to get too close to said animal. Just make sure YOU don't get too close to it.
\item Get to the habit of remembering your vehicles position [K] when [e]xamining it. The last thing you want to do is to search for your shopping cart that you have left behind because you were chased or you simply forgot.
\item Vehicular tool stations are off-limits to you? No Problem. A couple solar panels on your car, coupled with extra car batteries got you covered. Just install a Battery Compartment Mod into the most commonly used tools (Welder, Forge, Food Dehydrator, Food Processor) and load them up with car batteries.
\item Maximize butchering outputs! A Metal Butchering Rack is not difficult to make at the mid game and allows you to create a butchering station on the fly. Keep a couple 2x4's and nails on hand to construct a table, or make the Tourist Table for a deployable one. This also means having high butchering quality helps drastically - a Hunting Knife is the most powerful tool that does not require batteries.
\item It's headed right towards you? Yell at it [C]! Many animals are scared by loud noises and can therefore be deterred by yelling at them. This only works for natural wildlife, so deer(which are already afraid of humans) and bears - and even then it doesn't always work, but hey, it might just save your life. Wolves can not be feared by noise, and as of recently, the noise trigger for moose has been removed.
\end{itemize}

\vspace{\baselineskip}
To recap: do NOT be afraid to play carefully: [X] to peek around corners, [e]xamine closed windows that have their curtains pulled in order to peek out of them and run whenever you feel like you don't want to take a fight - there's no shame in surviving.